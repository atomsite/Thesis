\chapter{Background}

\section{Early-Type Stars}
\label{sec:earlytype}

The term Early-type stars is quite possibly the epitome of Astrophysical naming conventions, it's a very old term, coming from the dawn of astrophysics itself, quite opaque as to what it means, and also by definition \textit{completely wrong}. In fact it is one of the most wrong pieces of terminology I can think of.\footnote{Aside from astrophysicists calling something ``warm'', of course. That can quite literally mean anything from 10 to 10,000 Kelvin, depending on who you ask, what they're writing about, or how they're feeling at that particular moment. In fact, I'll probably end up falling into this same trap somewhere in this thesis as well!}
The first generation of astrophysicists found themselves asking very important questions such as ``what even \textit{are} stars'' and ``what possible mechanism can allow a star to burn for so long?'' Each of these questions was rather pressing for the burgeoning field, and the scientific community was aching for an answer.

Of course, like all pressing questions of the 19\textsuperscript{th} century, it fell to Lord Kelvin to provide a convincing but incorrect answer. Kelvin assumed that gravitational collapse was the mechanism for a stars long-term heating, with younger, ``early'' type stars shining the brightest. Not only was the mechanism incorrect, but typically older main sequence stars are more luminous than their younger counterparts of a similar mass! However, as is the case with astrophysical terminology, the term stuck, to the confusion of many young astrophysicists.


%//FIXME POTENTIAL maybe put this part in with OB stars? allows for more seamless switching from diatribe to the main body

Instead, we now know that stars produce their energy through fusion, from the substellar deuterium and lithium burning, to the low-mass pp-chain \& CNO hydrogen burning processes, and finally to the triple-$\alpha$ and more exotic fusion processes for evolved massive stars. The more massive the star the greater the internal pressure, leading to more difficult, but far more energetic fusion processes. This drastically increased energy output comes at the cost of a far shorter lifespan, as the rate of fuel consumption increases exponentially, while the available fuel has only increased a small amount \parencite{carrollIntroductionModernAstrophysics2014}.

\subsection{OB-type stars}
\label{sec:obtype}

And with that we shift our gaze to high-mass stars, with the most massive of all being the O and B type stars, these are extremely luminous ($\sim 10^4 \,\si{\solarluminosity}$), and relatively short lived ($\sim 10 \, \si{\mega\year}$) stars. The age-old adage of a candle burning twice as bright lasting half as long applies to our studies of the cosmos, but it is more apt to compare a candle and a stick of dynamite when considering stars on opposing ends of the Harvard classification system.

% Formation of OB stars, note binary systems!

The formation of massive stars is a significantly less understood phenomena than their low-mass counterparts. In the low-mass case, a giant cloud collapses, radiating energy, which lowers the radius of thermostatic equilibrium for the cloud, this collapse time is referred to as the Kelvin-Helmholtz timescale, $\tau_{KH}$. Another important timescale is the free-fall timescale, $\tau_{ff}$, which is the time taken for a cloud to collapse. 

%//FIXME above para needs work!

\begin{subequations}
  \begin{align}
      \tau_{KH} & \approx \frac{GM_*^2}{R_*L_*} \label{eq:khtime} ,\\
      \tau_{ff} & = \sqrt{\frac{3\pi}{32G\bar{\rho}}} \label{eq:fftime} ,
  \end{align}
  \label{eq:khfreefalltimes}
\end{subequations}

where $M_*$ is the protostellar mass, $R_*$ is the protostellar radius, $L_*$ is the protostellar luminosity, and $\rho$ is the mean density of the collapsing cloud.

Perhaps the most important distinction between massive star formation and its better understood counterpart is as a young protostar approaches the main sequence, the KH timescale is less than the free-fall timescale, meaning the material at the center of the collapsing cloud begins fusion while the bulk of core has collapsed onto the site of the future star. This burgeoning star begins to drive the weakly gravitationally coupled collapsing material away due to its sheer luminosity, driving this material outwards, causing it to accrete and shock material within the GMC. 

Another important consideration is the role of angular momentum as the star collapses.
The particularly massive cloud involved in massive star formation is more prone to fragmentation, meaning that massive stars typically form with an orbital partner, whilst approximately 2/3\textsuperscript{rds} of low-mass stars are part of a binary or multiple system, this value is near-total.
As such, the environment within an OB association after star formation consists of numerous young stars in tightly-knit groups disrupting the entire local area.\footnote{This is a bit like living in Headingley, Hyde Park, or any other area with lots of Undergraduates.}

%//TODO cite "near total" statistic, it's in one of the more recent papers you've had a look at, maybe a williams one?

Above a stellar mass of $1.3 \si{\solarmass}$ pressures and temperatures within a stellar core favour the fusion of hydrogen into helium through the catalytic CNO cycle, instead of the more direct p-p fusion process. 

\begin{subequations}
  \begin{align*}
    \prescript{12}{6}{C} + \prescript{1}{1}{H} & \rightarrow \prescript{13}{7}{N} + \gamma \\ 
    \prescript{13}{7}{N} & \rightarrow \prescript{13}{6}{C} + e^+ + \nu_e \\
    \prescript{13}{6}{C} + \prescript{1}{1}{H} & \rightarrow \prescript{14}{7}{N} + \gamma \\
    \prescript{14}{7}{N} + \prescript{1}{1}{H} & \rightarrow \prescript{15}{8}{O} + \gamma \\
    \prescript{15}{8}{O} & \rightarrow \prescript{15}{7}{N} + e^+ + \nu_e \\
    \prescript{15}{7}{N} + \prescript{1}{1}{H} & \rightarrow \prescript{12}{6}{C} + \prescript{4}{2}{He} 
  \end{align*}
\end{subequations}
%//FIXME sort out italics, use regular fonts

This mechanism is much more energetic than p-p fusion, with a far higher reaction rate that increases sharply as core temperature increases, resulting in a convective core, surrounded by a radiative envelope \parencite{salarisEvolutionStarsStellar2005}. This is the driving force behind the incredible luminosities of an OB star as it hurtles along the main sequence.

%//TODO might be a good idea to include that one graph with fusion rates?

% Influence of OB stars on surrounding environment, outsized influence etc.

% Touch on stellar winds, expand in next section

% Evolution and fate of massive stars, evolution, depletion of hydrogen products, then onto helium burning

Unfortunately for massive stars, pesky fundamental laws such as the conservation of energy come into play. With only an order of magnitude or two of additional mass more than our sun and shining $10^4$ times as brightly, this curtails the life of the brightest stars to lifespans not much more than $10^7$ years.
If we define a galactic year as the time it takes for a star to orbit the Milky Way, these poor stars don't even make it to their first birthdays, which is quite sad really.\footnote{Continuing this analogy our sun can drink, might have voted if they felt like it, and may be racking up vast quantities of student debt.}


As the available hydrogen begins to become depleted, the lowering reaction rates force the star to shrink, this raises the internal temperature until the core begins to burn helium through the triple-$\alpha$ process:

\begin{subequations}
  \begin{align*}
    \prescript{4}{2}{\text{He}} + \prescript{4}{2}{\text{He}} & \rightarrow \prescript{8}{4}{\text{Be}} \\
    \prescript{8}{4}{\text{Be}} + \prescript{4}{2}{\text{He}} & \rightarrow \prescript{12}{6}{\text{C}} + 2\gamma
  \end{align*}
\end{subequations}

The sudden spike in energy radiating from the core shifts the calculus of hydrostatic equilibrium in the favour of outward forces, causing the star to rapidly expand in the form of a Red Supergiant or Luminous Blue Variable \parencite{ryanStellarEvolutionNucleosynthesis2010a}.
During this phase the energy output of the star is even greater, with a timescale of $\sim 10^6$ years, this is only temporarily prolonging the life of the star, which will inevitably begin burning heavier and heavier elements, faster and faster.
Once the star starts producing iron its fate is sealed, the star stops fusing, and collapses, annihilating itself in the form of a supernova and leaving behind a remnant of its core in the form of a neutron star or black hole 
\parencite{ward-thompsonIntroductionStarFormation2011}.

Whilst the stars end is as inevitable as it is violent, the intermediate stage as the star leaves the main sequence is in itself extremely interesting, and for the context of this thesis, no product of this stage is more interesting than the Wolf-Rayet.



\subsection{Wolf-Rayet stars}
\label{sec:wrtype}


%Notes: Evolution type 
% Jumping off point, wiki wolf-rayet current models, research papers therein
% For WC formation O->RSG->WNE->WC 20-45 msol
% WR in general O->LBV->WR

% History and background of Wolf-Rayet stars

% Description of Wolf-Rayet star, introducing concept of strong stellar wind

As we now know, Wolf-Rayets are evolved forms of O-type stars, and are a short lived lived component, typically existing for around \num{5e5} years. Despite this relatively transient length of this stage, the influence of a WR star on its local medium is extremely outsized. The primary  

% Singlet and binary formation mechanisms
% Singlet -> LBV or RSG phase, contraction helium burning
% Binary  -> Common envelope evoluion 

Two cases for evolution from an O-type star to a WR exist, if a single star evolves as a massive star, it undergoes a 
\parencite{neugentWolfRayetContent2019}


\parencite{cherepashchukEvolutionWolfRayetStars2003}

\cite{crowther_physical_2007} details 

\begin{subequations}
  \begin{align*}
    \text{O} \rightarrow \text{LBV/RSG} \rightarrow \text{WN(H-poor)} \rightarrow \text{WC} \rightarrow \text{SN 1b} & ,~~ \text{if } 25 \, \si{\solarmass} < \text{M}_{\text{WR}} < 40 \, \si{\solarmass} \\
    \text{O} \rightarrow \text{LBV} \rightarrow \text{WN(H-poor)} \rightarrow \text{WC} \rightarrow \text{SN 1c} & ,~~ \text{if } 40 \, \si{\solarmass} < \text{M}_{\text{WR}} < 75 \, \si{\solarmass} \\
    \text{O} \rightarrow \text{WN(H-rich)} \rightarrow \text{LBV} \rightarrow \text{WN(H-poor)} \rightarrow \text{WC} \rightarrow \text{SN 1c} & ,~~ \text{if } \text{M}_{\text{WR}} > 75 \si{\solarmass} 
  \end{align*}
\end{subequations}

For lower mass stars this , however, in stars with extremely high masses, a WN type star can be initiated before 

In the case of a binary system however, as the more massive star begins to expand and the stars are sufficiently close, a common envelope can form between the \parencite{ibenCommonEnvelopesBinary1993}



\parencite{oswaltPlanetsStarsStellar2013}


%Subcategorising wolf-rayets into WN, WC and WO

Wolf-Rayet stars can be subcategorised through spectroscopic observation, which indicates enrichment in a particular element, the 3 major subtypes, WN, WC and WO are defined by their strong nitrogen, carbon and oxygen lines respectively.
While this subdivision was initially spectrographic in nature, these subtypes form differently, and there does not appear to be a continuous 
The important distinction between WN and WC/WO stars is that WN stars are enriched through hydrogen burning, whilst WC and WO are enriched through the by-products of helium burning \parencite{vinkVeryMassiveStars2015}.

% subcategorisation through numerical system

%Focus on WC, why they are so importanet 

For this thesis, only WC subtype Wolf-Rayets will be considered, as these are the only systems known to form dust, 


\section{Stellar Winds}
\label{sec:winds}

% Section covers wind driving mechanisms, while wind is touched on in the previous section, this covers the mechanisms more thoroughly

Stellar winds have already been discussed to some extent in the previous section, however, due to the significance of winds within this body of work, further detailing of winds must be discussed to gain a better understanding of the dynamics of Colliding Wind Binary systems. This section will cover in brief the study of stellar winds, particularly driving mechanisms from low and high mass stars.

% Background of stellar winds, particularly history of the subject

\subsection{Stellar winds in low mass stars}
\label{sec:lowmasswinds}

\subsubsection{Thomson scattering wind driving}

\subsubsection{Dust driven winds}

\subsection{Stellar winds in high mass stars}
\label{sec:radlinedriving}

\begin{table}[h]
  \centering
  % \resizebox{\textwidth}{!}{%
  \begin{tabular}{llll}
  \multicolumn{1}{c}{Star} & \multicolumn{1}{c}{$\dot M$} & \multicolumn{1}{c}{$v_\infty$} & \multicolumn{1}{c}{Mechanism} \\
  \multicolumn{1}{c}{}     & \multicolumn{1}{c}{$\si{\solarmass\per\year}$}         & \multicolumn{1}{c}{$\si{\kilo\metre\per\second}$}           & \multicolumn{1}{c}{}          \\ \hline
  Sun            & $10^{-14}$        & 400  & Thomson scattering \\
  Red Giant      & $10^{-7}-10^{-9}$ & 30   & Dust driven        \\
  Red Supergiant & $10^{-4}-10^{-6}$ & 10   & Dust driven        \\
  OB Star        & $10^{-7}-10^{-8}$ & 2500 & Line driving       \\
  Wolf-Rayet     & $10^{-5}$         & 1500 & Line driving       \\ \hline
  \end{tabular}%
  % }
  \caption{Comparison of winds from various types of star}
  \label{tab:windcomp}
\end{table}

\subsection{The CAK formalism}
\label{sec:cak}

\section{Interstellar Dust}
\label{sec:dust}

\subsection{The importance of interstellar dust}
\label{sec:dustimportance}

\subsection{Interstellar dust in massive star systems}
\label{sec:dustmassivestars}

\section{Colliding Wind Binary Systems}
\label{sec:cwb}

\subsection{The Wind Collision Region}
\label{sec:wcr}

% What is this region

% Brief notes on astrophysical shocks, link to appendix

%Detailed breakdown of Wind collision region

% Stagnation point

\subsection{Cooling in the WCR}
\label{sec:wcrcooling}

% Needs to be spun off into a subsection

% Radiative cooling, include graphs, mechanisms

% What constitutes 

% Dust cooling? Might need to move CWB dust formation up

\subsection{Dust formation in CWB systems}
\label{sec:cwbdust}

%Intro to dust formation in said systems

%Observational data Link to Williams papers in particular, dust formations only around WC

%Theories as to why

\subsection{Important WCd systems}
\label{sec:knowndustysystems}

