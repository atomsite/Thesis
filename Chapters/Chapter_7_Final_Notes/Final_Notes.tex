\chapter{Final Notes and Conclusions}

\section{Conclusions}

\subsection{Causes of dust formation in WCd systems}

\subsection{The role of eccentricity in dust formation}

\section{Future Study}

\subsection{More complex models}

\subsection{Further simulations of observed systems}

\subsection{Radiative transfer}

\subsection{WR+WR systems}

\subsection{Next generation telescopes}

\section{Other Observations}

\subsection[\textit{How I Learned to Stop Worrying and Love Numerics}]{\textit{Doctorate Strangelove or: How I Learned to Stop Worrying and Love Numerics}} 

\subsection{Join the physics department, see the world}

\subsection{Paul Erd\H{o}s was probably onto something}

\subsection{Carinae Strain: PhD research in a time of pandemic}

\subsection{WR 104 as a local GRB candidate}

\section{Research software acknowledgements}

This work was undertaken on ARC4 and ARC3, part of the High Performance Computing facilities at the University of Leeds, UK.

A good deal of the data reduction of this thesis was conducted using the Python 3 programming language \parencite{10.5555/1593511}, in particular, the following open source modules were used extensively:

\begin{itemize}
  \item \footlink{\texttt{NumPy}}{https://numpy.org/} \parencite{harris2020array}
  \item \footlink{\texttt{Astropy}}{https://www.astropy.org/} \parencite{astropy:2013,astropy:2018}
  \item \footlink{\texttt{Matplotlib}}{https://matplotlib.org/} \parencite{Hunter:2007}
\end{itemize}

\athena{} \parencite{athena} was also used extensively throughout the work in this thesis.

GNU Parallel \parencite{tange_2021_5523272} was used to speed up the batch processing of data within this project, if parallel programming is difficult, sometimes the only option is to run many, \textit{many} serial programmes at once.
Finally, this thesis was typeset with \LaTeX{}, using the {\TeX}live distribution and \texttt{latexmk} for compilation.
It is abundantly clear that scientists the world over owe an enormous debt of gratitude to Donald Knuth and Leslie Lamport for their work on the \TeX{} and \LaTeX{} projects.
Let's hope that the version $\pi$ update isn't coming too soon.