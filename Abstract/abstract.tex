
% Thesis Abstract -----------------------------------------------------


%\begin{abstractslong}    %uncommenting this line, gives a different abstract heading
\begin{abstracts}        %this creates the heading for the abstract page

\setlength{\parindent}{17.62482pt}
\setlength{\parskip}{0.0pt plus 1.0pt}

\note{Colliding Wind Binary (CWB) systems are relatively rare phenomena, but have a significant influence on galactic evolution in terms of dust production -- especially in the early universe.
The mechanisms behind this dust production, however, are poorly understood.
The strong winds from both partners in the binary system drive shocks that heat the dust forming region to temperatures in excess of \SI{1e8}{K}; whilst this region does rapidly cool, the initial shock temperatures would destroy any dust grains that formed outside the collision region.
Furthermore, this collision region is difficult to observe and simulate, limiting our understanding of how grains form and evolve in this region.}

\note{This thesis attempts to improve our understanding of the evolution of dust grains within these systems, particularly growth of these grains from small dust grain cores to micron-scale grains.
A co-moving dust grain model was implemented that simulates growth through accretion of gas onto the dust grains, as well as destruction through gas-grain sputtering.
The model also simulates cooling through collisional excitation and subsequent emission for both dust grains and gas.
Overall, the goal of this model was to determine how dust growth was influenced by the wind and orbital characteristics of the system, and which of these characteristics were most important for dust growth.}

\note{First, a parameter space exploration of dust producing CWB systems (WCd systems) was conducted, varying the orbital separation ($\dsep$), the wind terminal velocity ($\vinf$) and the mass loss rate ($\mdot$) of each star.
It was found that dust production is strongly influenced by the ratio of wind terminal velocities between each star, as well as the orbital separation.
Following up on this, a limited simulation of the episodic dust forming system WR140 was conducted, in order to understand how variance in orbital separation through eccentricity changed dust production rates over the course of a periastron passage.
Furthermore, it was determined that dust production occurs over a very short period immediately prior to periastron passage and a small period after, with an ``active'' phase of approximately 1 year, or $1/8$\textsuperscript{th} of the systems orbital period}

\note{Whilst there is much to be done in the future, and many more systems to be simulated (in particular the recently discovered WR+WR CWB systems WR48a and WR70-16) this model is a good first step towards shedding light on these elusive and dust-shrouded systems.}

\end{abstracts}
%\end{abstractlongs}

% ----------------------------------------------------------------------

%%% Local Variables: 
%%% mode: latex
%%% TeX-master: "../thesis"
%%% End: 
