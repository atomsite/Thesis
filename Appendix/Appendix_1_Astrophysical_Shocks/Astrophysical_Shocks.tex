\chapter{Astrophysical Shocks}

\chapter{Software Carpentry}

\section{Software Acknowledgements}

This work was undertaken on ARC4 and ARC3, part of the High Performance Computing facilities at the University of Leeds, UK.

A good deal of the data reduction of this thesis was conducted using the Python 3 programming language \parencite{10.5555/1593511}, in particular, the following open source modules were used extensively:

\begin{itemize}
  \item \footlink{\texttt{NumPy}}{https://numpy.org/} \parencite{harris2020array}
  \item \footlink{\texttt{Astropy}}{https://www.astropy.org/} \parencite{astropy:2013,astropy:2018}
  \item \footlink{\texttt{Matplotlib}}{https://matplotlib.org/} \parencite{Hunter:2007}
\end{itemize}

\athena{} \parencite{athena} was also used extensively throughout the work in this thesis.

Some diagrams in this work, specifically Fig. \ref{fig:quiver-h} and Fig. \ref{fig:quiver-oplus} use the \footlink{Quiver}{https://q.uiver.app/} communicative diagram editor.
GNU Parallel \parencite{tange_2021_5523272} was used to speed up the batch processing of data within this project, if parallel programming is difficult, sometimes the only option is to run many, \textit{many} serial programmes at once.
The \footlink{\texttt{Hyperfine}}{https://github.com/sharkdp/hyperfine} command line benchmarking tool was used to benchmark \athena{} throughout the project, in order to find ideal configuration parameters such as numerical integrators, core counts and meshblock sizes.


Most of the line plots in this project were produced by the 
The \footlink{\texttt{turbo}}{https://ai.googleblog.com/2019/08/turbo-improved-rainbow-colormap-for.html} palette is used for
A modified version of the \texttt{turbo.pal} palette file from the \footlink{\texttt{gnuplotting.org} GitHub repo}{https://github.com/Gnuplotting/gnuplot-palettes/blob/master/turbo.pal} was used to import the colour scheme into \texttt{gnuplot}.

Finally, this thesis was typeset with \LaTeX{}, using the {\TeX}live distribution and \texttt{latexmk} for compilation.
It is abundantly clear that scientists the world over owe an enormous debt of gratitude to Donald Knuth and Leslie Lamport for their work on the \TeX{} and \LaTeX{} projects.
Let's hope that the version $\pi$ update isn't coming too soon.
The thesis template is a modified version of the \footlink{Leeds Condensed Matter Physics Group \LaTeX{} template}{https://github.com/stonerlab/Thesis-template} which suited the needs of this thesis extremely well.

\section{Amdahl's Law}

\section{Version Control}